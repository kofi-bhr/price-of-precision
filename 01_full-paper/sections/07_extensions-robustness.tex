\section{Extensions and Robustness}

\subsection{Alternative Structures}
Our model's tractability relies on the additive bias structure. However, the core insight about informational motives for bias is robust to alternative specifications.

\paragraph{Multiplicative Bias.} A model with $s = \theta(1+bg) + \varepsilon$ yields qualitatively similar results. The Informativeness Principle still applies, but the quantitative effects and optimal threshold calculations become more difficult.

\paragraph{Feature Selection Bias.} If bias arises from including a feature that is predictive but also correlated with group status, the firm faces a trade-off between the information gained from the feature and the bias it induces. This creates an isomorphic problem structure.

\paragraph{Endogenous Group Inference.} If group membership `g` must be inferred from data (proxy discrimination), inference errors act as another source of noise, complicating but not eliminating the firm's trade-off.

\subsection{Market Dynamics}
Our static, single-firm model provides a foundation for understanding more involved market interactions.

\paragraph{Oligopoly Competition.} In an N-firm model of simultaneous bias choice, equilibrium outcomes depend on the nature of competition. If firms compete for the same pool of applicants, competition could create a ``race to the bottom" on fairness as firms seek any possible predictive edge. Conversely, if fairness can be used as a competitive advantage to attract talent, firms might differentiate their bias levels.

\paragraph{Candidate Responses.} Candidates are not passive. If a firm's bias $b^*$ becomes known, it can trigger strategic responses such as application decisions or attempts to ``game" the algorithm. These long-run effects could discipline the firm's initial choice of bias.