\section{Limitations and Future Directions}

Our model provides a tractable framework for isolating the informational motive for algorithmic bias, distinct from taste-based or statistical discrimination. The stylized structure, necessary for this theoretical clarity, naturally suggests several avenues for future research that can build upon this foundation by relaxing key assumptions.

\paragraph{Empirical Identification of the Fairness-Accuracy Frontier.}
A core contribution of our paper is to frame the firm's choice as an optimization problem constrained by a technological frontier, characterized by the parameter $\kappa$. Our central prediction is that variation in observed bias, $b^*$, is driven by variation in this underlying technological trade-off. The most pressing empirical challenge is therefore the identification of $\kappa$. A naive regression of outcomes on group status would conflate the firm's endogenous choice of bias ($b^*$) with the technological constraint ($\kappa$) it faces. A credible identification strategy would require either direct experimental data or a natural experiment that exogenously shifts the value of predictive accuracy or the cost of bias, allowing the researcher to trace out the frontier. For instance, a sudden increase in market competition could plausibly increase the marginal value of accuracy, inducing firms to move along their existing frontier and thus revealing its shape.

\paragraph{Endogenizing Costs and General Equilibrium Effects.}
In our model, the social costs of bias are captured by an exogenous function, $E(b)$. A significant theoretical extension would be to endogenize these costs within a dynamic framework. For example, persistent bias from incumbents (a high $b^*$) could discourage human capital investment by the disadvantaged group, as in \citep{CoateLoury1993}, leading to the very group-level productivity differences our static model assumes away. Furthermore, our single-firm analysis abstracts from general equilibrium effects. In a market setting, a firm's choice of bias could be a strategic complement or substitute to its rivals' choices. This could lead to a "race to the bottom" for predictive accuracy, or alternatively, create a market niche for a "fair" firm to attract talent, depending on the nature of competition and the observability of bias.

\paragraph{Strategic Candidates and Dynamic Learning.}
We model candidates as passive agents whose productivity is drawn from a fixed distribution. In reality, individuals may strategically alter their behavior in response to a known algorithm, a phenomenon known as "gaming." A richer model would incorporate a second stage where candidates react to the firm's choice of $b^*$. This could discipline the firm's initial choice; if a high bias is easily gamed, the firm may preemptively choose a lower $b$ to preserve the signal's integrity. Additionally, our firm is perfectly informed about the model's parameters. Future work could explore the dynamics of a firm learning about the shape of the fairness-accuracy frontier over time, potentially leading to path dependence or periods of suboptimal bias as it experiments.

\paragraph{Generalizing the Bias-Precision Trade-off.}
To maintain tractability, we model bias as a one-dimensional choice ($b$) affecting a single protected group, with a simple additive structure. Real-world applications involve a more complex problem space. Future theoretical work could model the trade-off along multiple dimensions, for instance, across multiple protected groups (race, gender) or multiple fairness constraints (e.g., demographic parity vs. equalized odds). This would transform the firm's problem from choosing a point on a line to selecting a point on a high-dimensional Pareto frontier. Furthermore, exploring alternative bias structures, such as multiplicative bias or bias arising from endogenous feature selection, would provide valuable insights into how the specific form of the algorithm's construction influences the nature of the trade-off and the firm's optimal policy.