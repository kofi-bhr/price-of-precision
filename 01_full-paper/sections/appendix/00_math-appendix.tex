\section{Mathematical appendix}

\subsection{Model setup and notation}

Before proving the main results, we establish the complete mathematical framework.

\subsubsection{Signal structure}
For a candidate from group $g \in \{0,1\}$ with true productivity $\theta \sim N(\mu, \sigma_\theta^2)$, the firm observes:
\begin{align}
s_g &= \theta + b \cdot \mathbf{1}_{g=1} + \varepsilon(b)
\end{align}
where $\varepsilon(b) \sim N(0, \sigma_\varepsilon^2(b))$ is a noise term with variance $\sigma_\varepsilon^2(b) = \sigma_0^2 + \kappa(b_{max} - b)$. Under our baseline assumption A1, we assume identical measurement across groups.

\subsubsection{Firm's optimization problem}
The firm chooses $b$ and $t$ to maximize the expected productivity of hired workers:
\begin{align}
V(b,t) &= \sum_{g \in \{0,1\}} \pi_g \int_t^\infty E[\theta | s, g, b] f(s|g,b) \, ds \label{eq:value_function_appendix}
\end{align}
For any given $b$, the optimal threshold $t^*(b)$ satisfies $E[\theta | t^*(b), g, b] = \mu$, where $\mu$ is the reservation value.

\subsection{Preliminary lemmas}

\begin{lemma}
At $b = 0$, the optimal threshold is $t^*(0) = \mu$.
\end{lemma}
\begin{proof}
At $b = 0$, both groups have identical signal distributions. A risk-neutral firm hires when $E[\theta|s,g,b] \geq \mu$. The threshold $t^*$ is where $E[\theta|t^*, g, b] = \mu$.
At $b=0$, the posterior is $E[\theta | t^*(0), g, 0] = \frac{\sigma_\theta^2 t^*(0) + \sigma_\varepsilon^2(0) \mu}{\sigma_\theta^2 + \sigma_\varepsilon^2(0)}$. Setting this to $\mu$ and solving gives $t^*(0)=\mu$.
\end{proof}

\begin{lemma}
The partial derivatives of the posterior mean are:
\begin{align}
\frac{\partial E[\theta | s, g, b]}{\partial b} &= -\frac{\sigma_\theta^2 \mathbf{1}_{g=1}}{\sigma_s^2(b)} + \kappa \frac{E[\theta|s,g,b] - \mu}{\sigma_s^2(b)} \\
\frac{\partial E[\theta | s, g, b]}{\partial \sigma_\varepsilon^2} &= \frac{\sigma_\theta^2(\mu - s + b \cdot \mathbf{1}_{g=1})}{(\sigma_s^2(b))^2}
\end{align}
\end{lemma}
\begin{proof}
These follow from applying the quotient rule to Equation \ref{eq:posterior_appendix} and noting that $\frac{\partial \sigma_\varepsilon^2}{\partial b} = -\kappa$.
\end{proof}

\begin{lemma}[Concavity of the Value Function]
The firm's value function $V(b)$ is strictly concave in $b$ for $b \in [0, b_{max}]$. That is, $\frac{d^2V}{db^2} < 0$.
\end{lemma}
\begin{proof}
The second derivative, derived using Leibniz's rule on the firm's value function, is the sum of a "threshold effect" and an "intramarginal effect." Both effects are negative due to the quadratic nature of the precision-bias trade-off, diminishing returns to signal precision, and the increasing marginal cost of signal distortion. Thus, $V''(b) < 0$.
\end{proof}

\subsection{Proof of Proposition 3: Existence of optimal bias ($b^* > 0$)}

We evaluate $\frac{dV}{db}$ at $b=0$. From Lemma 1, the hiring threshold is $t^*(0)=\mu$. The derivative consists of a negative distortion effect for group 1 and a positive precision effect for both groups.
\begin{align}
\text{Distortion} &= \pi_1 \, \mathbb{E}_{s_1}\left[-\frac{\sigma_\theta^2}{\sigma_s^2(0)} \mathbf{1}_{s_1 \ge \mu}\right] = -\frac{\pi_1 \sigma_\theta^2}{2\sigma_s^2(0)} \\
\text{Precision} &= \sum_g \pi_g \mathbb{E}_{s_g}\left[ \kappa \frac{E[\theta|s_g,g,0]-\mu}{\sigma_s^2(0)} \mathbf{1}_{s_g \ge \mu} \right] = \frac{\kappa \sigma_\theta^2}{\sigma_s^2(0)\,\sigma_s(0)\,\sqrt{2\pi}}
\end{align}
The derivative $\frac{dV}{db}\Big|_{b=0}$ is the sum of these two terms. It is positive if and only if:
\begin{align}
\kappa \geq \pi_1 \sigma_s(0) \sqrt{\tfrac{\pi}{2}} \;\equiv\; \kappa_{\min}
\end{align}
For any $\kappa > 0$, there is a corresponding $\kappa_{\min}>0$. Since $V(b)$ is concave (Lemma 3), a positive slope at $b=0$ implies the optimum $b^*$ must be strictly greater than zero.

\subsection{Proof of Proposition 4: Comparative static ($\frac{\partial b^*}{\partial \kappa} > 0$)}

The optimal bias $b^*$ satisfies the first-order condition $\partial V(b^*)/\partial b = 0$. To analyze how $b^*$ changes with $\kappa$, we use the Implicit Function Theorem. The conditions for the theorem are met: $V(b)$ is twice continuously differentiable in $b$ and $\kappa$ because it is constructed from integrals of Normal PDFs, which are smooth functions. Furthermore, the denominator in the resulting expression, $\partial^2 V / \partial b^2$, is strictly negative from Lemma 3, ensuring it is non-zero at the optimum. By the Implicit Function Theorem:
\begin{align}
\frac{\partial b^*}{\partial \kappa} = -\frac{\partial^2 V / \partial b \partial \kappa}{\partial^2 V / \partial b^2}\bigg|_{b=b^*}
\end{align}
From Lemma 3, the denominator is negative. The cross-partial derivative in the numerator, $\frac{\partial^2 V}{\partial b \partial \kappa}$, is positive because a larger $\kappa$ (a steeper trade-off) increases the marginal benefit of bias for all hired candidates. Therefore:
\begin{align}
\frac{\partial b^*}{\partial \kappa} = -\frac{(+)}{(-)} > 0
\end{align}

\subsection{Welfare analysis}

\subsubsection{Social welfare function}
The social planner maximizes $SWF(b) = V(b) - E(b)$, where $E(b) = \alpha b + \frac{\beta b^2}{2}$ represents the external costs of bias, with $\alpha > 0, \beta \ge 0$.

\subsubsection{Social optimum}
The social first-order condition is $\frac{dV}{db} - \alpha - \beta b = 0$. At the social optimum $b^{**}$, we have $\frac{dV}{db}\big|_{b=b^{**}} = \alpha + \beta b^{**} > 0$. Since the private optimum satisfies $\frac{dV}{db}\big|_{b=b^*} = 0$ and $V(b)$ is concave, it must be that $b^{**} < b^*$.

\subsubsection{Optimal pigouvian tax}
A tax $\tau$ per unit of bias leads to the firm's FOC: $\frac{dV}{db} - \tau = 0$. To induce the social optimum, the optimal tax $\tau^*$ must be set equal to the marginal external cost at $b^{**}$, so $\tau^* = \alpha + \beta b^{**}$.

\subsection{Extensions and robustness}

\subsubsection{Alternative functional forms}
Our results hold for a more general precision-bias trade-off, $\sigma_\varepsilon^2(b) = \sigma_0^2 + g(b_{max} - b)$, where $g(\cdot)$ is any increasing, differentiable function. As long as the derivative $g'(\cdot)>0$, the precision effect remains positive, and the core result ($b^*>0$) holds. The linear form is chosen for tractability.